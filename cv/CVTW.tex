%%%%%%%%%%%%%%%%%%%%%%%%%%%%%%%%%%%%%%%%%%%%%%%%%%%%%%%%%%%%%%%%%%%%%%%%
%%%%%%%%%%%%%%%%%%%%%% Simple LaTeX CV Template %%%%%%%%%%%%%%%%%%%%%%%%
%%%%%%%%%%%%%%%%%%%%%%%%%%%%%%%%%%%%%%%%%%%%%%%%%%%%%%%%%%%%%%%%%%%%%%%%

%%%%%%%%%%%%%%%%%%%%%%%%%%%%%%%%%%%%%%%%%%%%%%%%%%%%%%%%%%%%%%%%%%%%%%%%
%% NOTE: If you find that it says                                     %%
%%                                                                    %%
%%                           1 of ??                                  %%
%%                                                                    %%
%% at the bottom of your first page, this means that the AUX file     %%
%% was not available when you ran LaTeX on this source. Simply RERUN  %%
%% LaTeX to get the ``??'' replaced with the number of the last page  %%
%% of the document. The AUX file will be generated on the first run   %%
%% of LaTeX and used on the second run to fill in all of the          %%
%% references.                                                        %%
%%%%%%%%%%%%%%%%%%%%%%%%%%%%%%%%%%%%%%%%%%%%%%%%%%%%%%%%%%%%%%%%%%%%%%%%

%%%%%%%%%%%%%%%%%%%%%%%%%%%% Document Setup %%%%%%%%%%%%%%%%%%%%%%%%%%%%

% Don't like 10pt? Try 11pt or 12pt
\documentclass[12pt]{article}

% This is a helpful package that puts math inside length specifications
\usepackage{calc}
\usepackage{pifont}
\usepackage{marvosym}


% Simpler bibsection for CV sections
% (thanks to natbib for inspiration)
\makeatletter
\newlength{\bibhang}
\setlength{\bibhang}{1em}
\newlength{\bibsep}
 {\@listi \global\bibsep\itemsep \global\advance\bibsep by\parsep}
\newenvironment{bibsection}%
        {\vspace{-\baselineskip}\begin{list}{}{%
       \setlength{\leftmargin}{\bibhang}%
       \setlength{\itemindent}{-\leftmargin}%
       \setlength{\itemsep}{\bibsep}%
       \setlength{\parsep}{\z@}%
        \setlength{\partopsep}{0pt}%
        \setlength{\topsep}{0pt}}}
        {\end{list}\vspace{-.6\baselineskip}}
\makeatother

% Layout: Puts the section titles on left side of page
\reversemarginpar

%
%         PAPER SIZE, PAGE NUMBER, AND DOCUMENT LAYOUT NOTES:
%
% The next \usepackage line changes the layout for CV style section
% headings as marginal notes. It also sets up the paper size as either
% letter or A4. By default, letter was used. If A4 paper is desired,
% comment out the letterpaper lines and uncomment the a4paper lines.
%
% As you can see, the margin widths and section title widths can be
% easily adjusted.
%
% ALSO: Notice that the includefoot option can be commented OUT in order
% to put the PAGE NUMBER *IN* the bottom margin. This will make the
% effective text area larger.
%
% IF YOU WISH TO REMOVE THE ``of LASTPAGE'' next to each page number,
% see the note about the +LP and -LP lines below. Comment out the +LP
% and uncomment the -LP.
%
% IF YOU WISH TO REMOVE PAGE NUMBERS, be sure that the includefoot line
% is uncommented and ALSO uncomment the \pagestyle{empty} a few lines
% below.
%

%% Use these lines for letter-sized paper
%\usepackage[paper=letterpaper,
%           %includefoot, % Uncomment to put page number above margin
%            marginparwidth=0.7in,     % Length of section titles
%            marginparsep=.05in,       % Space between titles and text
%            margin=0.5in,               % 1 inch margins
%            includemp]{geometry}

% Use these lines for A4-sized paper
\usepackage[paper=a4paper,
            %includefoot, % Uncomment to put page number above margin
            marginparwidth=24mm,    % Length of section titles
            marginparsep=1mm,       % Space between titles and text
            margin=15mm,              % 25mm margins
            includemp]{geometry}

%% More layout: Get rid of indenting throughout entire document
\setlength{\parindent}{0in}

%% This gives us fun enumeration environments. compactitem will be nice.
\usepackage{paralist}
\usepackage[shortlabels]{enumitem}
% \usepackage[misc]{ifsym}
%% Reference the last page in the page number
%
% NOTE: comment the +LP line and uncomment the -LP line to have page
%       numbers without the ``of ##'' last page reference)
%
% NOTE: uncomment the \pagestyle{empty} line to get rid of all page
%       numbers (make sure includefoot is commented out above)
%
\usepackage{fancyhdr,lastpage}
\pagestyle{fancy}
%\pagestyle{empty}      % Uncomment this to get rid of page numbers
\fancyhf{}\renewcommand{\headrulewidth}{0pt}
\fancyfootoffset{\marginparsep+\marginparwidth}
\newlength{\footpageshift}
\setlength{\footpageshift}
          {0.1\textwidth+0.1\marginparsep+0.1\marginparwidth-2in}
\lfoot{\hspace{\footpageshift}%
       \parbox{3.5in}{\, \hfill %
                    \arabic{page} of \protect\pageref*{LastPage} % +LP
%                    \arabic{page}                               % -LP
                    \hfill \,}}

% Finally, give us PDF bookmarks
\usepackage{color,hyperref}
\definecolor{darkblue}{rgb}{0.0,0.0,0.3}
\hypersetup{colorlinks,breaklinks,
            linkcolor=darkblue,urlcolor=darkblue,
            anchorcolor=darkblue,citecolor=darkblue}

%%%%%%%%%%%%%%%%%%%%%%%% End Document Setup %%%%%%%%%%%%%%%%%%%%%%%%%%%%


%%%%%%%%%%%%%%%%%%%%%%%%%%% Helper Commands %%%%%%%%%%%%%%%%%%%%%%%%%%%%

% The title (name) with a horizontal rule under it
%
% Usage: \makeheading{name}
%
% Place at top of document. It should be the first thing.
\newcommand{\makeheading}[1]%
        {\hspace*{-\marginparsep minus \marginparwidth}%
         \begin{minipage}[t]{\textwidth+\marginparwidth+\marginparsep}%
                {\large \bfseries #1}\\[-0.15\baselineskip]%
                 \rule{\columnwidth}{1pt}%
         \end{minipage}}

% The section headings
%
% Usage: \section{section name}
%
% Follow this section IMMEDIATELY with the first line of the section
% text. Do not put whitespace in between. That is, do this:
%
%       \section{My Information}
%       Here is my information.
%
% and NOT this:
%
%       \section{My Information}
%
%       Here is my information.
%
% Otherwise the top of the section header will not line up with the top
% of the section. Of course, using a single comment character (%) on
% empty lines allows for the function of the first example with the
% readability of the second example.
\renewcommand{\section}[2]%
        {\pagebreak[2]\vspace{1\baselineskip}%
         \phantomsection\addcontentsline{toc}{section}{#1}%
         \hspace{0in}%
         \marginpar{
         \raggedright \scshape #1}#2}

% An itemize-style list with lots of space between items
\newenvironment{outerlist}[1][\enskip\textbullet]%
        {\begin{itemize}[#1]}{\end{itemize}%
         \vspace{-0.6\baselineskip}}

% An environment IDENTICAL to outerlist that has better pre-list spacing
% when used as the first thing in a \section
\newenvironment{lonelist}[1][\enskip\textbullet]%
        {\vspace{-\baselineskip}\begin{list}{#1}{%
        \setlength{\partopsep}{0pt}%
        \setlength{\topsep}{0pt}}}
        {\end{list}\vspace{-.6\baselineskip}}

% An itemize-style list with little space between items
% \newenvironment{innerlist}[1][\enskip\textbullet]%
\newenvironment{innerlist}[1][\enskip$\circ$]%
        {\begin{compactitem}[#1]}{\end{compactitem}}

% An environment IDENTICAL to innerlist that has better pre-list spacing
% when used as the first thing in a \section
\newenvironment{loneinnerlist}[1][\enskip\textbullet]%
        {\vspace{-\baselineskip}\begin{compactitem}[#1]}
        {\end{compactitem}\vspace{-.6\baselineskip}}

% To add some paragraph space between lines.
% This also tells LaTeX to preferably break a page on one of these gaps
% if there is a needed pagebreak nearby.
\newcommand{\blankline}{\quad\pagebreak[2]}

% Uses hyperref to link DOI
\newcommand\doilink[1]{\href{http://dx.doi.org/#1}{#1}}
\newcommand\doi[1]{doi:\doilink{#1}}


%%%%%%%%%%%%%%%%%%%%%%%% End Helper Commands %%%%%%%%%%%%%%%%%%%%%%%%%%%

%%%%%%%%%%%%%%%%%%%%%%%%% Begin CV Document %%%%%%%%%%%%%%%%%%%%%%%%%%%%

%\hyphenpenalty = 9999
\def\vs{\vspace{-0.1in}}
\begin{document}
% \makeheading{Curriculum Vitae\\ [0.3cm] TIEP HUU VU\quad~~~~~~\quad\quad\quad\quad\quad\quad\quad\quad\quad\quad\quad\quad\quad\quad{\small Last update: December 17, 2015}}
\makeheading{Tao Wang \hfill {\small Last update: \today}}


\section{Contact Information}

\newlength{\rcollength}\setlength{\rcollength}{3 in}
\vs
\begin{tabular}[t]{@{}p{\textwidth-\rcollength}p{\rcollength}}
Johns Hopkins University  & \texttt{Homepage:}\href{http://taowangecon.github.io}{http://taowangecon.github.io}\\
Wyman Park Building, 5th Floor & \texttt{GitHub:} \href{https://github.com/iworld1991}{https://github.com/iworld1991}\\
3100 Wyman Park Dr  &  {\large\Letter} \texttt{E-mail:}\href{mailto:twang80@jhu.edu}{twang80@jhu.edu} \\
Baltimore, MD 21211 &  Phone: (410) 516-7601
\end{tabular} 
%% ==============================================================
\vspace{0.2in}
\section{Research Interest} % (fold)
\label{sec:research_backg}
\vspace{-0.25in}
\begin{outerlist}
  \item {\bf Behavioral macroeconomics}: expectation formation
  \item {\bf Heterogeneous-agent macroeconomics:} household behaviors and macroeconomic dynamics 
\end{outerlist}
% section research_backg (end)
%% =========  ==============================
\section{Education}
    \href{https://www.jhu.edu}{\textbf{Johns Hopkins University}}, Baltimore, MD \hfill 2017-- 2022 (expected)
    \begin{outerlist}
        \item M.A. and Ph.D. candidate in \href{https://econ.jhu.edu} {Economics}
        \item Advisor: Prof. \href{https://econ.jhu.edu/directory/christopher-carroll/}{Christopher Carroll} 
\item Coursework: Time Series Econometrics, Asset Pricing, Decision Making under Uncertainty, Computational Macroeconomics, Information in Economics and Finance, International Finance, Advanced Macroeconomics I \& II
    \end{outerlist}
 \vspace{0.1in}
    \href{http://cornell.edu}{\textbf{Cornell University}}, Ithaca, NY \hfill 2013--2015
    \begin{outerlist}
        \item M.P.A. in \href{https://www.human.cornell.edu/cipa}{Cornell Institute for Public Affairs} 
        \item Thesis: \textit{Chinese Macroeconomic Policies, Restrictive Central Bank Independence, and Market Expectations}, advisor: Prof. \href{https://m.tau.ac.il/~razin/}{Assaf Razin}
    \end{outerlist}

\vspace{0.1in}
    \href{https://www.ruc.edu.cn/en}{\textbf{Renmin University of China}}, Beijing, China \hfill 2009--2013
    \begin{outerlist}
        \item B.A. in Economics 
        \item Thesis: \textit{Economic Behaviors of Local Governments}
    \end{outerlist}

%% ================== block:  ==========================
\section{Work in Progress}
\vspace{-.25in}
\begin{enumerate}
    \item \textit{``Rigidity of Expectations, Additional Evidence from Density Forecasts of the Inflation by Professionals and Households''}, \href{https://taowangecon.github.io/papers/InfVar.pdf}{working paper draft}, 2019.
\begin{innerlist}
\item[] \textit{Abstract}: Density forecasts of macroeconomic variables provide one additional moment restriction, uncertainty, for testing and exploring the implications of theories about how people form expectations differently from full-information rationality benchmark. This paper first documents the persistent dispersion in inflation uncertainty of professionals and households, and how it conveys different information from the widely used proxies to uncertainty such as cross-sectional disagreement and forecast errors. Second, utilizing the panel data structure of both surveys, I provide additional reduced-form test results as well as structural estimates for each particular theory of ``irrational expectation'' by jointly accounting for its predictions for different moments. This is a natural extension of Coibion and Gorodnichenko (2012), which examines different moments separately. Also, motivated by the time-varying pattern of the uncertainty observed from surveys, I extend their work to allow for an alternative inflation process featuring stochastic volatility. These extensions allow me to match the joint dynamics of inflation and forecast moments in a better degree of fitness. It also illustrates how incorporating higher moments from survey data helps understand both the expectation formation mechanisms and inflation dynamics.   
\end{innerlist}
\item  \textit{``Perceived Income Risks''}, \href{https://taowangecon.github.io/papers/PerceivedIncomeRisk.pdf}{work in progress}.
\begin{innerlist}
\item[] \textit{Abstract}: What econometricians have assumed to be the labor income risks facing agents based on estimates from cross-sectional inequality may not necessarily be consistent with what is truely perceived.  This work in progress studies individual-specific perceived income risks utilizing a density survey of labor income. Empirically, I have found the earners who are female, from low-income households and low education has higher perceived risks. And both perceived variance and tail risk measure skewness are negatively correlated with stock market returns. The ongoing work includes characterizing how the subjective differences are both driven by the true income risk profile and perceptual heterogeneity from alternative mechanisms of expectation formation deviating from full-informational rationality. I will also incorporate empirical findings in an otherwise standard life-cycle model of consumption and portfolio choice to explore their implications on consumption insurance and asset pricing.
\end{innerlist}
\end{enumerate}   

%% =========  ==============================
\section{Research Assistant Experience} % (fold)
\label{sec:research_exper}
\vspace{-0.25in}
\begin{outerlist}
    \item {\bf Heterogeneous-agent macroeconomic modeling} \hfill 2019\\
   \href{https://github.com/econ-ark}{Econ-ARK Project} led by Prof. Christopher Carroll
    \begin{innerlist}
      \item Wrote \href{https://github.com/econ-ark/HARK/blob/master/HARK/BayerLuetticke/notebooks/DCT-Copula-Illustration.ipynb}{Python code} that evaluates, illustrates and visualizes the dimension-reduction by Bayer and Luetticke (2018).
\item Contributed to the \href{https://github.com/econ-ark/DemARK/blob/master/notebooks/KrusellSmith.ipynb}{Jupyter notebook} replicating Krusell-Smith (1998) algorithm via \href{https://github.com/econ-ark/HARK}{HARK} toolkit. 
    \end{innerlist}

\item {\bf Tutorials on quantitative economic modeling } \hfill 2017 \\
\href{https://quantecon.org}{Quantitative Economics}, Prof. Thomas Sargent (New York University)


\item {\bf Capital account liberalization and exchange rate dynamics} \hfill 2015 \\
Prof. Eswar Prasad (Cornell University)
\end{outerlist}


\section{Employment Experiences}
\vspace{-0.25in}
\begin{outerlist}
\item \textbf{International Monetary Fund} \hfill 2016-2017 \\
 Research Assistant in the Research Department
\item \textbf{The Brookings Institution} \hfill 2015-2016 \\
 Research Analyst in Global Economy and Development Program
\end{outerlist}

%% =========  ==============================
\section{Technical Skills} % (fold)
\label{sec:technical_ski}
\vspace{-0.25in}
\begin{outerlist}
  \item {\it Programming Languages}: Python, Matlab, Stata
\end{outerlist}


%% ================== block:  ==========================
\section{Policy Publication}
\vspace{-.25in}
\begin{enumerate}
    \item {\bf A Sy and T Wang }. \textit{``De-risking, Renminbi Internationalization and Regional Integration: Evidence from Payment Flows of Sub-Saharan Africa''.} \href{https://www.brookings.edu/research/de-risking-renminbi-internationalization-and-regional-integration/}{Brookings Working Paper}, 2016.
\end{enumerate} 
%% ==============================================================
\section{References}
\begin{itemize}
%\item[]\def\halfblankline{\vspace{0.1in}}
\item Prof. \href{http://www.econ2.jhu.edu/people/ccarroll/index.html}{\textbf{Christopher Carroll}} (JHU), \href{mailto:ccarroll@jhu.edu} {ccarroll@jhu.edu}

\item Prof. \href{https://econ.jhu.edu/directory/jonathan-wright/}{\textbf{Jonathan Wright}} (JHU),  \href{mailto:wrightj@jhu.edu} {wrightj@jhu.edu}
\end{itemize}
\end{document}